\documentclass[a4paper,12pt]{report}
\usepackage{geometry}
\usepackage[T1]{fontenc}
\usepackage[utf8]{inputenc}
\usepackage{lmodern}
\usepackage{setspace}
\usepackage{musicography}
\usepackage{times}
\geometry{a4paper, portrait, margin=1in}

\title{Analysis of J.S. Bach's Prelude in C Major}
\author{Vincent Santini}

\begin{document}

\maketitle

\begin{abstract}
\doublespacing
J.S. Bach's Prelude in C Major is the first of several exercises in \textit{The Well Tempered Clavier}. It's often played by beginner pianists as an exercise in harmony and chord structure, and was intended by Bach for "\textit{the use of musical youth desirous of learning, and especially for the pastime of those already skilled in this study}". However, it wasn't originally written with the piano in mind, as the piano only rose in popularity following Bach's death. Instead, Bach wrote the Prelude and other exercises to be played by any keyboard instrument that was widely used at the time: the harpsichord, the clavichord, or even the pipe organ.\\
\indent Though the progressions remain simple through much of the piece, the chords themselves develop in interesting ways. Much of the harmonic tension is carried through use of secondary dominant and leading tone chords and is resolved quickly, a common practice for the time, as this dissonance must follow particular rules. However, by the end of the piece, Bach prolongs the tension much more so, developing certain chords in strange ways for the time. This could perhaps have been meant as a final challenge for the aspiring keyboardist to put their ear to the test when approaching atypical chord progression.
\end{abstract}


\section*{A Progression in Four Parts}
\doublespacing
The Prelude opens with a simple harmonic idea:
\begin{center}
I $\rightarrow$ ii\musFig{4 2} $\rightarrow$ V\musFig{6 5} $\rightarrow$ I \linebreak
Tonic $\rightarrow$ Predominant $\rightarrow$ Dominant $\rightarrow$ Tonic
\end{center}

\doublespacing
This is the most basic occurence of this idea, and it becomes the underlying format of the rest of the piece's progressions, but as the piece continues, we are shown more complex methods of prolonging this basic idea. Bach demonstrates this immediately by modulating into the dominant key of G Major, following the first Imperfect Authentic Cadence with a pivot chord on vi\musFig{6}, jumping to V\musFig{4 2} in G, and then resolving to I\musFig{6} to ground us in G Major.\\
\doublespacing
\indent Later, Bach introduces the first secondary leading tone chord following a return to C Major. It's not subtle either, the broken chord structure he uses throughout the piece helps emphasize the dissonance of the chord, and even more so with the \fl 7 sustained in the tenor voice. It's a stark contrast to the stable, though inverted 7th chords used before, and it only lasts a measure before resolving to the supertonic ii\musFig{6}. This section is interesting in that Bach mirrors this same vii\textdegree $\rightarrow$ I resolution immediately after in C Major:

\begin{center}
$\frac{vii^{\circ}\musFig{4 3}}{ii}$ $\rightarrow$ ii\musFig{6} $\rightarrow$ vii\textdegree\musFig{4 3} $\rightarrow$ I \linebreak
\end{center}


\section*{Harmonic Oddities}
\doublespacing
The first instance where Bach begins to diverge from expected harmonic progression is a little more than halfway through the piece. After resolving a secondary leading tone chord similar to the one above in a major subdominant IV\musFig{7}, Bach sets up another secondary leading tone vii\textdegree\musFig{7} over the dominant V. This would be expected to resolve to V in C Major, and until now, that is exactly how Bach has dealt with these particular chords. However, instead, this secondary dominant "resolves" directly to the diminished seventh of C Major:

\begin{center}
$\frac{vii^{\circ}\musFig{7}}{V}$ $\rightarrow$ vii\textdegree\musFig{4 2}
\end{center}

\doublespacing
This could be interpreted as simply another method of harmonic prolongation, as Bach follows this strange resolution with a V\musFig{7}, the chord we should have expected to come a measure earlier, and then ends the phrase with another Imperfect Authentic Cadence. \\
\indent This strange progression occurs one more time in the piece only five measures later with a similarly odd flow:

\begin{center}
V\musFig{7} $\rightarrow$ $\frac{vii^{\circ}\musFig{4 2}}{V}$ $\rightarrow$ I\musFig{6 4}
\end{center}

\doublespacing
Once again Bach prolongs the tension of this building coda, and I believe these two oddities serve a more subtle purpose. Resolving from such a distant chord to what is supposed to be our tonic- not to mention in the \musFig{6 4} inversion- also disorients our ears. We may have resolved to our "home base" on the tonic in C Major, but it certainly doesn't \textit{sound} like it. This effect is compounded by the droning pedal G Bach sustains in the bass for much of this piece's ending; it teases the listener by moving in and out of chords where G is a chord tone. Bach is preparing something with this harmonic confusion, and he is extremely clever in the way he resolves it.\\
\indent With the pedal G held through the I\musFig{6 4} and ii\musFig{7}, it is finally grounded and recontextualized under a dominant V\musFig{7} chord. This sets the piece up to finally find its footing on C, but Bach sneaks one more complication in before allowing a more satisfying resolution:

\begin{center}
I\musFig{6 4} $\rightarrow$ ii\musFig{7} $\rightarrow$ V\musFig{7} $\rightarrow$ $\frac{V\musFig{7}}{IV}$ $\rightarrow$ IV\musFig{6 4}
\end{center}

\doublespacing
The V\musFig{7} progresses to another secondary dominant! This seems strange at first, but because $\frac{V\musFig{7}}{IV}$ is actually a C Major seventh chord, it \textit{does} provide some sense of ground, just with an odd "flavor" provided by the \fl 7 due to F Major's key signature. Resolving this into IV\musFig{6 4} constructs the predominant of the final harmonic progression, followed by the dominant V\musFig{7} and concluded with a Perfect Authentic Cadence on the tonic, the only PAC in the whole piece. \\
\indent The final progression is fundamentally the same as those before it. With the Tonic disguised as the dominant of IV, the piece concludes with a more complex character, but remains as structurally simple as it began:

\begin{center}
$\frac{V\musFig{7}}{IV}$(I\musFig{\fl 7}) $\rightarrow$ IV\musFig{6 4} $\rightarrow$ V\musFig{7} $\rightarrow$ I \linebreak
Tonic $\rightarrow$ Predominant $\rightarrow$ Dominant $\rightarrow$ Tonic
\end{center}


\end{document}
